\begin{figure*}
  \centering
  \begin{subfigure}[b]{0.42\textwidth}
    \centering
    \includegraphics[width=\textwidth]{binder/binder/steady-vs-tilted/teeplots/annotation-size-bits=64+differentia-width-bits=1+downsample=500+hue=policy+num-generations=100000+population-size=65536+post=teed-figure-subplots-adjust-right-0-6-teed-set-titles-row-template-row-name.../+row=variable+score=value+viz=peckplot+x=scenario+x-group=outer+y=value+ext=}
    \caption{Example reconstruction quality distributions. Lower is better.}
    \label{fig:steady-vs-tilted-summary-example}
  \end{subfigure}%
  \begin{subfigure}[b]{0.58\textwidth}
    \centering
    \includegraphics[width=\textwidth]{binder/binder/steady-vs-tilted/teeplots/col=metric+hue=relative-grade+kind=strip+post=teed-set-titles-col-template-col-name+row=policy+viz=catplot+x=mean-rank+y=scenario+ext=}
    \caption{Reconstruction quality comparison outcomes. Lower is better.}
    \label{fig:steady-vs-tilted-summary-overview}
  \end{subfigure}
  \caption{%
    \textbf{How does retention policy affect reconstruction quality?}
    Subpanel \ref{fig:steady-vs-tilted-summary-overview} shows mean rank among reconstruction error measures from tilted, hybrid, and steady retention policies across sensitivity analysis conditions.
    Color coding indicates significant outcome (Kruskal-Wallis H then Mann-Whitney U test).
    Lower is better.
    Tilted policy (top row) performs best in most evolutionary scenarios, except triplet distance under the highly phylogenetically-rich drift regime.
    Steady policy (bottom row) performs worst in most scenarios, except triplet distance under the drift regime.
    Hybrid policy performance is somewhat less than tilted policy, but is robust to the drift regime.
    Subpanel \ref{fig:steady-vs-tilted-summary-example} shows reconstruction quality effects for 64-bit size, bit-differentia annotations with population size 65,536, downsample size 500, and 100k generations.
    Background hagching indicates significant outcome.
    See Supplementary Figure \ref{fig:steady-vs-tilted} for listing of effects by sensitivity analysis condition.
  }
  \label{fig:steady-vs-tilted-summary}
\end{figure*}
