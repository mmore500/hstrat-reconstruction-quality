\section{Introduction} \label{sec:introduction}

Scaling up is an important thing for alife.
There are a lot of practical considerations, one of them being familiarity with the programming model and the tools getting more complicated.
In order to drive the field toward that direction, we need to have tools and have people know how to use them.
Specifically, tools should support what other people want to do, not just be an off the shelf system (although Avida is like that and is a good thing.)

One tool in this vein is hereditary stratigraphy methodology.
It solves the problem of phylogenetic tracking at scale.
This is important because phylogenies can help you do X, Y and Z.

However, it has not been known (and certainly has not been explained) the best way to configure it.
Here, we provide a curated, action-oriented introduction to hereditary stratigraphy with scenario-specific guidance on how to set it up.
We include new experiments that test which set up parameters matter and what choices are best.
By the end of this article, the reader will: (then list some learning objectives).

There is a good deal of existing software, and more n the way, so it should be fairly accessible.
However, new algorithms are data structure free and are more easy to implement in a one-off in a custom language or framework you need.
