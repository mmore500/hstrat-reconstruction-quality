\section{A Practicioner's Guide to Hereditary Stratigraphy} \label{sec:synthesis}

This section shifts focus from an analytical, hypothesis-driven framing, as applied in the previous section, to discussion that is instead prescriptive and summative.
The goal here is to synthesize results from reported annotation-and-reconstruct experiments, as well as other work testing and applying hereditary stratigraphy, to provide concrete, action-oriented advice to guide the reader in effectively applying hereditary stratigraphy in their digital evolution experiment, or other use case.

Discussion covers the following questions,
\begin{enumerate}
\item Should I use hereditary stratigraphy or phylogenetic direct tracking?
\item How should I choose appropriate hereditary stratigraphy configuration for my use case?
\item How do I integrate hereditary stratigraphy instrumentation into my digital evolution simulation?
\item How do I work with hereditary stratigraphy annotation data once it is generated?

\subsection{Hereditary Stratigraphy or Direct Tracking?}

If you are not using parallel or distributed computing, direct phylogenetic tracking should usually be preferred due to its capability for perfect record-keeping; likewise, if your simulation uses a centralized controller-worker paradigm.
A tool like Phylotrack (asexual phylogenies; Python/C++), APoGET (sexual phylogenies; C++) or MABE (C++) might be appropriate for your use case \citep{dolson2024phylotrack,bohm2017modular}.
It is also reasonably straightforward to implement phylogeny tracking yourself \citep{moreno2024algorithms}.
However, for serial or centralized simulations there are a few scenarios where reconstruction-based tracking may be useful,
\begin{itemize}
\item resource-constrained runtime environments where memory is scarce or dynamic memory allocation is not supported (e.g., embedded);
\item simulation objectives require hard real-time operations; or
\item \textit{ad hoc} serialization and re-use of agents across simulation runtimes, where maintaining a cohesive global record is difficult or impractical.
\end{itemize}

In simulations employing decentralized parallel and distributed computation, a reconstruction-based approach such as hereditary stratigraphy is more likely to be appropriate.
Compared to perfect tracking in this setting, hereditary stratigraphy provides simpler implementation, lower runtime communication overhead, and greater robustness to data loss.
Note that, if annotation size is not a concern or a low generation count is anticipated, essentially perfect-quality reconstruction can be achieved with hereditary stratigraphy methods.
In this case, one could simply use a retention policy that discards no differentia and a wide enough differentia width to effectively guarantee collisions will not occur (e.g., 32 or 64 bits).
However, in most cases, a compromise between phylogeny approximation and per-genome annotation size will be necessary.
We cover this in the next section.

\subsection{Algorithm Configuration}
\begin{itemize}
\item \textbf{Retention policy: steady or tilted?}
  If you suspect \textit{very weak} phylodiversity-enhancing factors (ecology, spatial structure, low seletion pressure), use a steady policy.
  If you expect strong phylodiversity-enhancing factors or are unsure, use a hybrid policy or a tilted policy.
  In rare cases where phylodiversity-enhancing factors are very strong (e.g., pure drift conditions), a tilted policy may be appropriate.
  Suggested default choice: hybrid policy.
\item \textbf{Differentia size: bit or byte?}
  Bit-size differentia maximize the fraction of correct reconstruction outcomes, but can also introduce incorrect reconstruction outcomes.
  Byte-size differentia have very low incorrect reconstruction outcome rates, but have a larger incidence of unresolved reconstruction oucomes (false polytomies).
  If you need very strong guarantees against incorrect reconstruction outcomes, an even larger differentia size (32 or even 64 bits) may be appropriate.
  Suggested default choice: bit-size differentia.
\item \textbf{Annotation size: constant or dynamic?}
  In most scenarios, a constant-size annotation will be more computationally efficient and ensure fuller use of available memory resources.
  However, if you need hard guarantees on recency-proportional inference quality, use a annotation size that scales $\mathcal{O}log(n)$ with generational depth.
  Suggested default choice: constant-size annotations.
\item \textbf{Annotation size: how many bits?}
  This factor trades off between memory-use and communication-bandwidth overhead for annotations and quality of reconstructed phylogenies.
  In cases where annotation size is not a limiting factor, when using single-bit differentia 256 bit annotations will discern phylogenetic events with about 13\% recency-relative precision (tilted) or 1\% depth-relative precision (steady) through 1 billion generations.
  For full-byte differentia, an annotation size on the order of kilobits would be very robust.
  Where annotation resource use is a limiting factor, a 64 bit annotation can give good results.

  Current implementations of surface algorithms are limited to buffer sizes that are even powers of two (32, 64, 128, etc.).
  Where finer gradations are desired, intermediate differentia sizes (e.g., storing 3 bit differentia over 32 surface sites would occupy 96 bits) or alternating depositions onto amalgamated surfaces might be considered (e.g., a 32 bit tilted surface and a 16 bit steady surface) --- although the latter would require some minor algorithm implementaiton customizations akin to those conducted for the hybrid surface.
  Column algorithms are more flexible in buffer sizing, but make less full use of available buffer space.

  According to results here, simulation scale does not appear to be a major factor in considering annotation size, because at the same absolute sample sizes samples from differently sized populations have no clear trend in reconstruction quality characteristics.
  However, sample size may play into decision of annotation size (see below).

  Note that in addition to differentia values, a generation counter will also need to be stored in genomes (for most use cases, a 32- or 64-bit value).

  In scenarios where explicitly differentiating between founding clades is paramount, consider adding a systematically assigned founder ID or randomly generated fixed tag.
  This tag would then be as a first pass to divvy end genomes between founding origins before feeding into the rest of the reconstruction pipeline.
  \footnote{One possible exception is cases where a global monotonic counter is available and it is desirable to demarcate phylogenetic history in terms of simulation time rather than generations.}
\item \textbf{Implementation: column or surface?}
  If you are using dynamic annotation size, you will need a column-based implementation to allow differentia count fluxuations.
  Otherwise, for constant annotation size, surface implementations are much more efficient \citep{TODOOTHERPAPER}.
  In the case of tilted retention policy, they also give higher-quality reconstructions.
  If using steady policy, column implementation gives higher-quality reconstructions and is preferable if computational performance is not a concern.
  Another factor that might influence this decision is the software platform being designed for.
  If packages providing hereditary stratigraphy are not available, surface algorithms are easier to implement owing to greater orthogonality between the algorithm implementaiton (site selection) and data structure (simple buffer).
  Suggested default choice: surface implementation.
\item \textbf{Administration: update frequency?}
  For most applications, hereditary stratigraph updates should correspond directly to generations elapsed.
  However, in cases with expected very high generation counts that outstrip the capacity of on-genome counter size and fine-resolution visibility over recent history is not necessary.
  In other circumstances, where time-indexed resolution is preferred or ultrametricity (even branch lengths from root) is paramount, it may be preferred to update annotations on the basis of simulation update cycles rather than generations.
  Under this approach, annotations on all genomes
  Suggested default choice: every-generation updates.
\item \textbf{Simulation content: sexual or asexual?}
  The core of existing hereditary stratigraphy is designed around asexual lineages and that makes it very straightforward and simple to apply to evolution simulations with asexual reproduction.
  Although there has been some preliminary work demonstrating applications of hereditary stratigraphy to sexual populations, operating procedures are not as well developed for this mode of evolution.
  One possibility includes tracking asexual lineages of individual genes (``gene trees'').
  Organism-level annotation could be used to track the emergence of new independently-breeding subpopulations (``species trees''').
  This approach relies on distinct annotation values reaching fixation within independent subpopulations.
  This can be accomplished through drift where offspring inherit the differentia record of a randomly-selected parent or, alternatively, through a gene drive mechanism where, for many-bit differentia, large-magnitude values are favored for inheritance and thereby rapdily sweep through interbreeding subpopulations.
  Refer to \citep{moreno2024methods} for additional discussion.
\item \textbf{Sampling strategies}
Care will need to be taken to account for sample size and sample size relative to the whole population in analyses.
Triplet distance appears to be largely stable under surveyed increases in sample size.
However, loss of inner nodes occuring near the tips is observed when increasing sampling size.
Where this is a problem, differentia count and/or size should be increased.

Note that reconstruction techniques are fully compatible to stitch together across widely-varying time points.
Indeed, these taxa can provide valuable information about extincted lineages.
Intermediate taxa can sould serve a role akin to ``fossils'' in studies of natural history.

For more advanced analyses, it may be desirable to associate sampled strata with trait data \citep{TODOMODES,TODOCITEFROMJACOB}.

\end{itemize}

\subsection{Runtime Integration}

Three major steps will be necessary to be implemented to run the hereditary stratigraphy pipeline: (1) integration of instrumentation into the simulation runtime genome model, (2) serialization of evolved agents' markers from simulation, (3) loading data into the Python \textit{hstrat} library in order to analyze it.

\begin{itemize}
\item \textbf{Runtime Annotations}:
You will need to augment the \texttt{Genome} data structure in your simulation code in two ways: (1) to store hereditary stratigraphic annotation data and (2) add a call to the copy/reproduce or mutate function to update the differentia store and bump the generation counter.

For sufaces, two components are required: a fixed-buffer differentia store and a generation counter.
The differentia store can be implemented as an array of integer data types (e.g., \texttt{uint8}), but for bit differentia you will likely want to use a raw memory array or an abstraction around it if available (e.g., \texttt{std::bitset}).

Surfaces also have a simple update procedure.
Call \texttt{pick\_deposition\_site} and randomize the differentia element at the returned nth position.

Software implementing column-based approaches is available for Python \citep{moreno2022hstrat} and surface-based approaches are available for Python \citep{TODO}, and Zig/Cerebras Software Language \citep{TODO}.
Based on community feedback, C/C++ and Rust are priority targets to port hstrat surface implementations to.

If you are using another software language, porting core algorithms should be doable with moderate effort.
Are implemented primarily as a library of small functions ($\approx<30$ LOC) with accompanying tests, making them amenable to stepping-stone piece by piece transliteration.
We found LLM assistance to be highly effective in translating the surface algorithms from Python to Zig \citep{TODO}.
Feel free to get in touch --- we would be happy to furnish these algorithms in other languages if you have a use case.
\end{itemize}

\subsection{Postprocessing and Analysis}
\begin{itemize}
\item \textbf{Serialization}
In order to transfer marker data for postprocessing, you will need to save it to file.
For best compatibility with the analysis pipeline, extract annotations and save them together in a conventional format separate from other genome components (e.g., JSON, CSV).
For each annotaiton, you will need to store the generation counter and the differentia data.
One convenient plain text way to encode differentia data is as a hex string.

In order to conduct postprocessing, you will need to use tools in the \textit{hstrat} library.
When reserializing the data, you will need to know the 11¡retention policy and differentia width used so make sure these are recorded somewhere.
Functions to convert raw data into intantiated \texttt{Column} objects are described in package documentation and examples, with a variety of entry points compatible with JSON, YAML, and CSV formats.

Note that, while the hstrat library also provides support to deserialize from compact binary formats in many circumstances, storage in plain text format with zipping (e.g., gzip) will provide competitive --- if not better --- space efficiency to binary representations and can be considerably easier to work with.

\item \textbf{Analysis}
Reconstruction algorithms are currently implemented in the \textit{hstrat} library.
This algorithm takes a list of deserialized column objects and produces a phylogeny.
Optionally, a list of taxa identifiers may also be provided if desired.
A variety of tools for estimation of MRCA between marker pairs or groups are also provided.
Reconstructions are reported in alife community data standard format \citep{TODO}.
Tools are provided that can convert these formats to standard bioinformatics formats for interoperation with the rich existing software ecosystem for phylogenetic visualization and analysis \citep{TODOalifedataphyloinformaticsconvert}.
