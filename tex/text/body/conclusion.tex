\section{Conclusion} \label{sec:conclusion}

In this work, we have applied empirical annotate-and-reconstruct experiments to benchmark reconstruction quality of hereditary stratigraphy approaches across use cases varying in phylogenetic structure, scale, and allocated annotation space.
In these experiments, we consider,
\begin{itemize}
\item differentia retention, whether annotation space should be allocated for finer resolution in discerning recent phylogenetic events and
\item annotation implementation, comparing existing column-based approaches to newer surface-based approaches optimized for fixed-size annotations.
% \item differentia width, how many bits should be used per lineage checkpoint to reduce the probability of overestimating relatedness.
\end{itemize}

Findings are then applied to develop practitioner-oriented guidelines to effectively employ hereditary stratigraphy methodology.
Principal takeaway results are,
\begin{enumerate}
\item tilted retention outperforms steady retention, except in scenarios with very high phylogenetic richness (e.g., drift conditions);
\item hybrid tilted-steady retention provides good reconstruction quality across scenarios;
\item for tilted retention, surface-based implementation provides better reconstruction quality than column-based implementation; and
\item for steady retention, column-based implementation provides better reconstruction quality than surface-based implementation.
% \item increased differentia size increases accuracy but reduces precision.
\end{enumerate}
As tilted policy is likely to be preferred in practice, it is promising to see surface-based implementation improve reconstruction quality in this case.
Because surface-based approaches were designed foremost to optimize performance and be easier to code for new platforms (particularly in low-level environments) \citep{moreno2024trackable}, additionally achieving enhanced reconstruction quality makes their adoption a win-win situation.

% Owing to its inspiration from inference-based phylogenetics work in biology, hereditary stratigraphy is designed to operate in an entirely decentralized manner that is by nature efficient to scale and robust to disruptions or data loss.
% It is therefore promising to see that reconstruction accuracy of hereditary stratigraphy is also generally robust to scale-up.
% On the other hand, we found inner node loss --- a precision measure --- to be sensitive to increases in the number of taxa sampled for reconstruction.
% This issue, with bit-width differentiae, arises due to increased probability for exactly identical annotations through differentia collision, resulting in clumping of tip nodes into polytomies.
% It is possible that this problem would be abated in systems with nonsynchronous generations, where tips would be spread apart by generational depth.
% That said, we did find inner node loss to be largely robust to scale-up of the actual population size of a simulation, with the number of taxa sampled for reconstruction held constant.

Our goal in developing hereditary stratigraphy is to provide methodology that is sufficiently lightweight, modular, and flexible for general-purpose use across digital evolution systems.
% Here, we have provided a comprehensive, evidence-driven foundation for effective application of hereditary stratigraphy across experimental use cases.
% Explicitly compiling this material as a prescriptive guide maximizes its utility to this end.
% However, we anticipate that --- most of all --- adoption hinges on success in providing a seamless, plug-and-play developer experience to those wishing to incorporate the methodology.
As such, we seek to provide packaged library software with easy-to-learn API design and thorough documentation.
Note that, beyond content presented here, the \textit{hstrat} repository includes a small library of code samples demonstrating end-to-end use of hereditary stratigraphy, useful as a starting point for new users \citep{moreno2022hstrat}.
We would be very interested in collaborating to integrate hereditary stratigraphy instrumentation into your system or to develop algorithm implementations for your particular programming language and runtime environment.

Present work motivates several further steps in developing hereditary stratigraphy methodology.
From a practical perspective, we wish to make improvements in curating public-facing surface-based implementations that are interoperable with existing column-based tools.
Another practical consideration will be optimization, and perhaps parallelization, of reconstruction to support work with very large taxon sets.
In a separate vein, accuracy loss from differentia collisions when working with bit-level differentia may warrant effort in developing means to sample among possible collision sets and generate a consensus tree with accompanying uncertainty measures \citep{bryant2003classification}.

Considering a broader perspective on future work, development of hereditary stratigraphy comprises only one aspect of a broader agenda in scaling up digital evolution experiments.
% These methods are only one part of working with larger, less-directly observable sytstems.
Among other avenues, research will need to explore simulation synchronization schemes \citep{fujimoto1990parallel}, best-effort computing approaches \citep{moreno2022best,ackley2020best}, emerging hardware architectures \citep{moreno2024trackable,chan2018lenia,heinemann2008artificial}, and scalable assays for evolutionary innovation, ecological dynamics, and various forms of complexity \citep{bedau1998classification,dolson2019modes,moreno2024methods,moreno2024case,moreno2024ecology}.
