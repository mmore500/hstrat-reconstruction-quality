\begin{abstract}
Computer simulations are an important tool for studying the mechanics of biological evolution.
In particular, \textit{in silico} work with agent-based models provides an opportunity to collect high-quality records of ancestry relationships among simulated agents.
Such phylogenies can provide insight into evolutionary dynamics within these simulations.
Existing work generally tracks lineages directly, yielding an exact phylogenetic record of evolutionary history.
However, direct tracking can be inefficient for large-scale, many-processor evolutionary simulations.
An alternate approach to extracting phylogenetic information from simulation that scales more favorably is \textit{post hoc} estimation, akin to how bioinformaticians build phylogenies by assessing genetic similarities between organisms.
Recently introduced ``hereditary stratigraphy'' algorithms provide means for efficient inference of phylogenetic history from non-coding annotations on simulated organisms' genomes.
A number of options exist in configuring hereditary stratigraphy methodology, but no work has yet tested how they impact reconstruction quality.
To address this question, we surveyed reconstruction accuracy under alternate configurations across a matrix of evolutionary conditions varying in selection pressure, spatial structure, and ecological dynamics.
We synthesize results from these experiments to suggest a prescriptive system of best practices for work with hereditary stratigraphy, ultimately guiding researchers in choosing appropriate instrumentation for large-scale simulation studies.
\end{abstract}
